%! Author = Mads Sørensen
%! Date = 14-11-2021

\section{Analyse}
Analysen af opgaven tager udgangspunkt i følgende artefakter; domænemodel, kravspecificering, use case diagram, use case beskrivelser, systemsekvensdiagram, sekvensdiagram.

%! Author = Mads Sørensen
%! Date = 11-11-2021

\section{Kravspecifikation}
Krav er opstillet ud fra den stillet opgave. Vi identificerede først kravene ud fra spilreglerne, som kan kategoriseres som funktionelle krav. Vi havde også ikke-funktionelle krav, som krav til dokumentation, konfiguration osv. Vi havde i projektet stort fokus på at fuldføre de krav, der bragt os tættest på det ønsket produkt.

\begin{table}[H]
   \renewcommand{\arraystretch}{1.5}
   \centering
   \caption{Funktionelle krav} \label{tabel:funkrav}
   \vspace{0.2cm}
   \begin{tabular}{ |p{0.5cm}|p{8cm}|p{2cm}| }
       \hline
       \textbf{ID} & \textbf{Krav} & \textbf{Prioritering} \\
       \hline
       01 & Spillet skal understøtte 2 til 4 spillere & Høj \\
       \hline
       02 & Hver spiller skal starte med $35 & Høj\\
       \hline
       03 & Hver spiller skal starte på startfeltet & Høj\\
       \hline
   \end{tabular}
 \end{table}

\subsection{Use case diagram}
\begin{figure}[H]
    \includegraphics[width=10cm]{figures/usecasediagram}
    \caption{Use case diagram}
\end{figure}

\subsection{Use cases}
Vi har identificeret og analyseret en række use cases i den udleverede opgave. De vigtigste use cases er: \textit{Start spil, Tag tur} og \textit{Træk chancekort}.

\begin{itemize}
    \item \textbf{UC1:} Start spil
    \begin{itemize}
        \item \textbf{UC1.1:} Vælg antal spillere
        \item \textbf{UC1.2:} Tag en tur
    \end{itemize}
    \item \textbf{UC2:} Tag en tur
    \begin{itemize}
        \item \textbf{UC2.1:} Betal/Modtag penge
        \item \textbf{UC2.2:} Ryk brik
        \item \textbf{UC2.3:} Rul terning
    \end{itemize}
    \item \textbf{UC3:} Træk chancekort
    \begin{itemize}
        \item \textbf{UC3.1:} Brug chancekort
    \end{itemize}
\end{itemize}


\textbf{Brief}

\begin{table}[H]
    \renewcommand{\arraystretch}{1.5}
    \centering
    \begin{table}{l{12cm}}
        \hline
        \textbf{ID: UC1} Start spil \\
        \hline
        \textbf{Actor:} Spiller\\
        \hline
        \textbf{Basic flow:} Spiller starter spillet.\\
        \hline
    \end{table}\\
    \begin{table}{l{12cm}}
        \hline
        \textbf{ID: UC1.1} Vælg antal spillere \\
        \hline
        \textbf{Actor:} Spiller\\
        \hline
        \textbf{Basic flow:} Spiller vælger hvor mange spillere der deltager i spillet.\\
        \hline
    \end{tabular}
\end{table}

        ID: UC1.2 Vælg brik \\
        Actor: Spiller \\
        Basic flow: Hver spiller vælger en brik som de får tildelt resten af spillet. \\
        ID: UC2 Tag en tur \\
        Actor: Spiller \\
        Basic flow: Spiller påbegynder tag en tur. \\
        ID: UC2.1 Betal/modtag penge \\
        Actor: Spiller \\
        Basic flow: Spiller betaler en modspiller (eller bank) penge. \\
        ID: UC2.2 Ryk brik \\
        Actor: Spiller \\
        Basic flow: Spiller rykker sin brik antal felter frem på brættet med det antal øjne, terningen viser. \\
        ID: UC2.3 Rul terning \\
        Actor: Spiller \\
        Basic flow: Når det er den pågældende spillers tur, så ruller spilleren terningen, hvorefter en terningsværdi fås. Værdien benyttes til at ændre spillerens position. \\
        ID: UC3 Træk chancekort \\
        Actor: Spiller \\
        Basic flow: Spiller trækker et chancekort, hvorefter et tilfældigt kort modtages. \\
        ID: UC3.1 Brug chancekort \\
        Actor: Spiller \\
        Basic flow: Spiller benytter sit chancekort. \\
        \bottomrule

