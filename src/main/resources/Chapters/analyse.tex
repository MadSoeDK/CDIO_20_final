%! Author = Mads Sørensen
%! Date = 14-11-2021

\section{Analyse}
Analysen af opgaven tager udgangspunkt i følgende artefakter; domænemodel, kravspecificering, use case diagram, use case beskrivelser, systemsekvensdiagram, sekvensdiagram.

%! Author = Mads Sørensen
%! Date = 11-11-2021

\section{Kravspecifikation}
Krav er opstillet ud fra den stillet opgave. Vi identificerede først kravene ud fra spilreglerne, som kan kategoriseres som funktionelle krav. Vi havde også ikke-funktionelle krav, som krav til dokumentation, konfiguration osv. Vi havde i projektet stort fokus på at fuldføre de krav, der bragt os tættest på det ønsket produkt.

\begin{table}[H]
   \renewcommand{\arraystretch}{1.5}
   \centering
   \caption{Funktionelle krav} \label{tabel:funkrav}
   \vspace{0.2cm}
   \begin{tabular}{ |p{0.5cm}|p{8cm}|p{2cm}| }
       \hline
       \textbf{ID} & \textbf{Krav} & \textbf{Prioritering} \\
       \hline
       01 & Spillet skal understøtte 2 til 4 spillere & Høj \\
       \hline
       02 & Hver spiller skal starte med $35 & Høj\\
       \hline
       03 & Hver spiller skal starte på startfeltet & Høj\\
       \hline
   \end{tabular}
 \end{table}

\subsection{Use case diagram}
\begin{figure}[H]
    \includegraphics[width=10cm]{figures/usecasediagram}
    \caption{Use case diagram}
\end{figure}

\subsection{Use cases}
Vi har identificeret og analyseret en række use cases i den udleverede opgave. De vigtigste use cases er: \textit{Start spil, Tag tur} og \textit{Træk chancekort}.

\begin{itemize}
    \item \textbf{UC1:} Start spil
    \begin{itemize}
        \item \textbf{UC1.1:} Vælg antal spillere
        \item \textbf{UC1.2:} Tag en tur
    \end{itemize}
    \item \textbf{UC2:} Tag en tur
    \begin{itemize}
        \item \textbf{UC2.1:} Betal/Modtag penge
        \item \textbf{UC2.2:} Ryk brik
        \item \textbf{UC2.3:} Rul terning
    \end{itemize}
    \item \textbf{UC3:} Træk chancekort
    \begin{itemize}
        \item \textbf{UC3.1:} Brug chancekort
    \end{itemize}
\end{itemize}
\newline
\textbf{Brief}\\
\newline
\textbf{ID: UC1} Start spil \\
\textbf{Actor:} Spiller\\
\textbf{Basic flow:} Spiller starter spillet.\\
\newline
\textbf{ID: UC1.1} Vælg antal spillere\\
\textbf{Actor:} Spiller\\
\textbf{Basic flow:} Spiller vælger hvor mange spillere der deltager i spillet.\\
\newline
\textbf{ID: UC1.2} Vælg brik \\
\textbf{Actor:} Spiller \\
\textbf{Basic flow}: Hver spiller vælger en brik som de får tildelt resten af spillet. \\
\newline
\textbf{ID: UC2} Tag en tur \\
\textbf{Actor:} Spiller \\
\textbf{Basic flow: } Spiller påbegynder tag en tur. \\
\newline
\textbf{ID: UC2.1} Betal/modtag penge \\
\textbf{Actor: } Spiller \\
\textbf{Basic flow} Spiller betaler en modspiller (eller bank) penge. \\
\newline
\textbf{ID: UC2.2} Ryk brik \\
\textbf{Actor: } Spiller \\
\textbf{Basic flow} Spiller rykker sin brik antal felter frem på brættet med det antal øjne, terningen viser. \\
\newline
\textbf{ID: UC2.3} Rul terning \\
\textbf{Actor: } Spiller \\
\textbf{Basic flow} Når det er den pågældende spillers tur, så ruller spilleren terningen, hvorefter en terningsværdi fås. Værdien benyttes til at ændre spillerens position. \\
\newline
\textbf{ID: UC3} Træk chancekort \\
\textbf{Actor: } Spiller \\
\textbf{Basic flow} Spiller trækker et chancekort, hvorefter et tilfældigt kort modtages. \\
\newline
\textbf{ID: UC3.1} Brug chancekort \\
\textbf{Actor: } Spiller \\
\textbf{Basic flow: } Spiller benytter sit chancekort. \\
\newline
\textbf{Fully dressed}
\begin{table}[H]
    \renewcommand{\arraystretch}{1.5}
    \centering
    \begin{tabular}{|p{14cm}|}
        \hline
        \textbf{ID:} UC2 Tag en Tur\\
        \hline
        \textbf{Actor:} Spiller\\
        \hline
        \textbf{Stakeholders}. Spillerne: Vil gerne få en høj balance og vinde spillet. Vil gerne underholdes.\\
        \hline
        \textbf{Pre-conditions:}\\
        1. Spillet skal være startet\\
        2. Det skal være den nuværende spillers tur\\
        \hline
        \textbf{Post-conditions:}\\
        1. Spillerens tur er slut eller spillet er vundet \\
        \hline
        \textbf{Basic flow} \\
        1. Spiller modtager besked fra GUI’en: “Spiller X tur, klik på rul terning” \\
        2. Spiller klikker på knap som ruller med terninger med resultat “X”\\
        3. Spillerens brik rykkes “X” antal felter frem\\
        4. Spillet tjekker hvilket felt de er landet på\\
        5. Feltet er ikke købt \\
        \hspace{0.2cm}5.1.1: Derfor betaler spilleren for feltets pris\\
        \hspace{0.2cm}5.1.2: Feltet opdateres til at være ejet af spilleren\\
        6. Spilleren passerede ikke start \\
        7. Spillet tjekker om spilleren har en negativ account balance \\
        7.1.1: Spilleren har en positiv balance \\
        8. Spillet ændre hvilken spiller er aktiv \\
        \hline
        \textbf{Alternative Flows}
        5.2.1: Feltet er købt \\
        5.2.2: Spilleren betaler feltets lejepris til ejeren af feltet \\
        5.3.1: Feltet er Chancen \\
        5.3.1.1: Spilleren trækker et chancekort \\
        5.3.1.2: Spilleren udfører chancekortets effekt \\
        5.4.1: Feltet er enten “Water show” eller “Fireworks”\\
        5.4.1.1: Spilleren betaler \$2 til “Loose Change” \\
        5.5.1: Feltet er “Loose Change” \\
        5.5.1.1: Spilleren tilføjer \$ lig med værdien af “Loose Change” \\
        5.5.1.2: “Loose Change” bliver sat lig med 0 \\
        5.6.1: Feltet er “Railroad” \\
        5.6.1.2: Spilleren ruller med terningen igen \\
        5.6.1.2. Spillet tjekker feltet (se basis flow 4) \\
        5.7.1: Feltet er “Gå i fængsel” \\
        5.7.1.1: Spilleren betaler \$3 \\
        5.7.1.2: Spillerens position bliver sat til “Fængsel” \\
        5.8.1: Feltet er “Fængsel” \\
        \hline
    \end{tabular}
\end{table}
\begin{table}[H]
    \renewcommand{\arraystretch}{1.5}
    \centering
    \begin{tabular}{|p{14cm}|} \\
        5.8.1.1: Spilleren betaler ingen penge \\
        6.1.1: Spilleren passerede startfeltet \\
        6.1.1: Spilleren indsamler \$2 \\
        7.2.1: Spilleren har en negativ balance \\
        7.2.1.1: Spillet tjekker hvilken spiller har den højeste balance \\
        7.2.1.2: En enkel spiller har højeste balance \\
        7.2.1.3: Den givne spiller vinder spillet og stopper spillet \\
        7.2.1.1: Flere spillere har den højeste balance \\
        7.2.2.2: De spillere har begge vundet og stopper spillet \\
        \hline
    \end{tabular}
\end{table}

